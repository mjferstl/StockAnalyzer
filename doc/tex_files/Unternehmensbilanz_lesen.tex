
%
%%
%%%
%%%%
\chapter{Unternehmensbilanzen lesen}

Die börsennotierten Unternehmen berichten ihre Ergebnisse überlicherweise jedes Quartal  (\englishName{10-Q}) und legen am Ende eines Geschäftsjahres auch eine Jahresbilanz (\englishName{10-K}) vor.
Die wichtigsten Informationen finden sich dabei in drei verschiedenen Blättern, die nachfolgend näher betrachtet werden.

%
%%
%%%
\section{Balance Sheet}
Das \englishName{Balance Sheet} gibt einen Überblick über das Vermögen und die Schulden eines Unternehmens und beschreibt somit den aktuellen finanziellen Zustand der Firma.
Wie der Name andeutet, befinden sich die Zahlen in diesem Blatt im Gleichgewicht (\textit{balance}), da folgendes Gleichgewicht gilt:
\formulaNote{\formulaText{Assets} - \formulaText{Liablilities} = \formulaText{Equity}}


%
%%
\subsection{Asset Accounts}

\begin{description}
    \item[Current Assets] sind die Vermögenswerte, die innerhalb eines Finanzzykluses (normalerweise 1 Jahr) zu Geld gemacht werden.
        Diese umfassen \englishName{cash and equivalents}, \englishName{short-term investments}, \englishName{accounts receivable} und \englishName{inventories}.
        \begin{description}
            \item[Cash and Equivalents and short-term-Investments] beschreibt, was das Unternehmen kurzfristig an Geld verfügbar hat.
            \item[Accounts Receivable] umfasst das Geld, das das Unternehmen noch nicht erhalten hat, aber erwartet demnächst zu erhalten.
                Wenn ein Produkt bereits an den Kunen ausgeliefert wurde, dessen Zahlung aber noch nicht eingegangen ist, dann wird der in der Kasse noch fehlende Betrag hier addiert.
                Wenn dieser Posten deutlich schneller wächst als der Umsatz, dann verbucht das Unternehmen viele Einnahmen, obwohl es dafür noch kein Geld erhalten hat.
                Dies kann zu Problemen führen, wenn das Unternehmen einen Teil der erwarteten Einnahmen gar nicht erhält.
                Manchmal findet sich im \englishName{Balance Sheet} auch ein Posten mit der Bezeichnung \englishName{allowance for doubtful accounts}, der den Betrag beschreibt, der dem Unternehmen zwar geschuldet wird, den das Unternehmen aber nicht mehr erwartet vom Kunden tatsächlich zu erhalten.
            \item[Inventories] umfasst Kapital, das in Form von Materialien, teilweise fertigen Produkten, hergestellten aber noch nicht verkauften Produkten etc. vorhanden ist.
                Der angegebene Betrag muss immer mit Vorsicht betrachtet werden, da die hierzu gezählten Produkte und Materialien zwar den angegebenen Preis haben können, es aber nicht sicher ist, dass bei einem plötzlichen Verkauf dieser Produkte und Materialien auch dieser Preis erzielt werden kann.
                Wenn ein Unternehmen weniger Material und fertige Produkte \glqq herumliegen \grqq\ hat, dann könnte es profitabler sein, da es weniger Kapital in diesen Vermögenswerten gebunden hat.
                Eine Aussage hierzu liefert der Kennwert \textbf{\englishName{inventory turnover}}, der angibt, wie oft das gesamte Inventar während eines Finanzzykluses umgeschlagen wurde.
                \begin{formel}
                    \formulaText{inventory turnover} = \frac{\formulaText{cost of goods sold}}{\formulaText{inventories}}
                \end{formel}
        \end{description}
    \item[Asset Accounts] sind langfristige Vermögenswerte, die nicht innerhalb eines Finanzzykluses zu Geld gemacht werden.
        Hierzu zählt Folgendes:
        \begin{description}
            \item[PP\&E Property, Plant and Equipment] umfasst das Kapital, das in Fabriken, Gebäuden, Grundstücken, Maschinen etc. gebunden ist.
                Wenn man diese Zahl mit dem Wert der gesamten Vermögenswerte (\englishName{Total Assets}) vergleicht, dann kann man ein Gefühl bekommen, wie kapitalintensiv das Unternehmen ist. 
            \item[Investments] ist das Kapital, das in langfristigen Anleihen und Aktien anderer Unternehmen steckt.
                Wenn dies einen signifikaten Anteil des Gesamtvermögens eines Unternehmens ausmacht, dann sollte man heraufinden, wo dieses Geld steckt, wie sicher es ist und wie lange es gebunden ist. 
            \item[Intangible Assets] beschreibt alle immateriellen Güter. Hierzu zählt z.B. auch der sog. \englishName{Goodwill}. 
                Dieser beschreibt das Geld, das bei einer Übernahme des Unternehmens notwendig ist und den Sachwert der Firma übersteigt, da z.B. auch der Ruf und die Marke \glqq gekauft\grqq\ werden.   
                Generell sollte dieser Wert mit Skepsis betrachtet werden, da Firmen den Wert generell sehr hoch ansetzen und es nicht sicher ist, dass dieser Betrag auch gerechtfertigt ist.
        \end{description}
\end{description}

%
%%
\subsection{Liabilitiy Accounts}

Dieser Anteil umfasst alle Verflichtungen (Schulden) des Unternehmens.
Diese werden in zwei Rubriken unterteilt.

\begin{description}
    \item[Current Liabilities] Geld, das das Unternehmen innerhalb eines Finanzzykluses ausgeben wird. 
        Dies umfasst folgende Posten.
        \begin{description}
            \item[Accounts Payable] Geld, das das Unternehmen Anderen schuldet, z.B. in Form offener Rechnungen. 
                Wenn das Unternehmen viele Rechnungen erst später bezahlt und das Geld deshalb noch in der Firma bleibt, dann hat dies positive Auswirkungen auf den \englishName{cash flow}. 
            \item[Short-Term-Borrowings] kurzfristig geliehenes Geld, das innerhalb eines Jahres zurückgezahlt werden muss. 
                Wenn das Unternehmen viel Geld kurzfristig geliehen hat, aber nicht so viel Geld kurzfristig verfügbar hat, dann kann das Unternehmen kurzfristige Geldprobleme haben.
                siehe auch \englishName{current ratio} in Kapitel \ref{sec:current_ratio}.
        \end{description}
    \item[Noncurrent Liabilities] Geld, das das Unternehmen nicht innerhalb des eines Finanzzykluses ausgibt und deshalb langfristige Verbindlichkeiten beschreiben.
        \begin{description}
            \item[Long-Term-Debt] Geld, das das Unternehmen z.B. der Bank schuldet, aber nicht in nächster Zeit zurückzahlen muss. 
        \end{description}
\end{description}

%
%%
\subsection{Stockholder's equity}

Dieser Posten umfasst das Eigenkapital des Unternehmens, das in Form von Aktien im Umlauf ist.
Die meisten der aufgeführten Unterpunkte haben eine geringe Relevanz.
Der einzige Eintrag, der hier einen Blick wert ist, ist unter dem Namen \englishName{Retained Earnings} aufgeführt.
\begin{description}
    \item[Retained Earnings] Kapital, das das Unternehmen über seine gesamte Lebensdauer generiert hat, abzüglich Dividendenzahlungen und Aktienrückkäufen, da dieses Geld zurück an die Aktionäre gewandert ist.
        Jedes Jahr, in dem das Unternehmen Gewinn macht und diesen nich vollständig an die Aktionäre ausschüttet, steigt dieser Betrag.
        Wenn das Unternehmen bisher kein Geld behalten sondern sogar verloren bzw. mehr ausgegeben hat, dann hat dieser Eintrag einen negativen Wert.
        Generell kann hier geprüft werden, wie das Unternehmen Geld über die Jahre einnimmt.
\end{description}

%
%%
%%%
\section{Income Statement}
Dieses Blatt beinhaltet die Einnahmen und Ausgaben innerhalb eines Finanzzykluses und beschreibt somit die aktuelle Entwicklung des Unternehmens.

\begin{description}
    \item[Revenue/Sales] Umsatz der Firma innerhalb eines Finanzzykluses.
        Große Unternehmen brechen den Umsatz auch oft auf einzelne Produkte, Geschäftsbereiche oder Regionen auf der Erde herunter.
        Firmen können ihren Umsatz aber zu verschiedenen Zeitpunkten verbuchen, da z.B. ein Softwarehersteller hier einen großen Geldbetrag vermerken kann, wenn ein Produkt an den Kunden ausgeliefert wird, wohingegen eine Service-Firma den Umsatz über die gesamte Vertragslaufzeit gleichmäßig verteilt.
    \item[Cost of Sales] Geld, das aufgewendet werden musste um den aufgeführten Umsatz zu generieren.
        Manchaml aus unter dem Namen \englishName{Cost of goods sold} aufgeführt. 
    \item[Gross Profit] enspricht der Differenz aus Umsatz und den Aufwendungen für den Umsatz und beschreibt das Geld aus dem gesamten Umsatz, das nicht für die Bereit- und Herstellung der Produkte erforderich ist.
        \begin{formel}
            \formulaText{Gross Profit} = \formulaText{Revenue} - \formulaText{Cost of Sales}
        \end{formel}
        Der prozentuale Anteil wird auch als \englishName{Gross margin} bezeichnet. 
        \formulaNote{\formulaText{Gross margin} = \frac{\formulaText{Gross Profit}}{\formulaText{Revenue}}} 
    \item[Selling, General and Administrative Expenses (SG\&A)] Ausgaben für Marketing, Verwaltung etc.
        Bei einer Firma mit großen Vertriebsnetzwerk, das viele Beschäftigte im Vertrieb hat ist dieser Posten oftmals groß, wobei diese Unternehmen meist aufgrund der höheren erzielten Produktepreise auch eine hohe \englishName{gross margin} haben.
    \item[Depreciation and Amortization] Abschreibungen von Gebäuden etc.
    \item[Nonrecurring Charges/Gains] Einmalzahlungen und -ausgaben.
        Dieser Posten sollte in der Regel einen geringen Anteil im Vergleich zum Umsatz darstellen.
        Falls ein Unternehmen im \englishName{Income Statement} immer wieder nicht-unwesentiche Beträge als Einmalzahlungen aufführt, dann sollte man hellhörig werden!
        Es könnte sein, dass die Firma dubiose Ausgaben vermehrt als Einmalzahlungen deklariert.
    \item[Operating Income] Unternehmensergebnis. Dieser Wert wird auch genutzt, um die \englishName{operation margin} zu ermitteln, die eine Grundlage für den Vergleich verschiedener Firmen und Industrien bildet.
        \begin{formel}
            \formulaText{Operating Income} = \formulaText{Revenues} - \left( \formulaText{Cost of Sales} + \formulaText{Cost of Sales}\right)
        \end{formel}
    \item[Interest Income/Expense] Ausgaben für bzw. Einnahmen aus Anleihen. 
        Manchmal werden die Einnahmen und Ausgaben auch unter dem Punkt \englishName{net interest income} verrechnet.
    \item[Taxes] Steuern.
        Wenn dieser Wert über die Jahre stark schwankt, dann versucht das Unternehmen möglicherweise mit Steuertricks Geld zu sparen. 
        Es ist deshalb möglich, dass die Firma das \glqq gesparte\grqq\ Geld irgendwann noch nachzahlen muss.
    \item[Net Income] Gewinn nach Abzug aller Ausgaben.
        Dieser Wert wird meistens von den Unternehmen hervorgehoben, jedoch sollte man ihn mit Vorsicht genießen, da er durch Einmahlzahlungen und Kapitalerträge verzerrt werden kann.
    \item[Number of Shares] Anzahl der im Umlauf befindlichen Aktien, die zur Berechnung des Gewinns pro Aktie genutzt werden.
        Hier gibt es verschiedene Arten
        \begin{description}
            \item[basic] beschreibt die Anzahl der tatsächlich im Umlauf befindlichen Aktien. Dieser Wert kann aber ignoriert werden und stattdessen der nachfolgende betrachtet werden.
            \item[diluted] verwässterte Anzahl an Aktien, die auch Optionen, die in Aktien umgewandelt werden können, wandelbare Anleihen (\englishName{convertible bonds}) und Ähnliches berücksichtigt.
                Dieser Wert gibt an, auf welchen Anteil am gesamten Unternehmen der eigene Anteil in Form erworbener Aktien sinken kann. 
        \end{description}
    \item[Earnings per Share] (EPS) Gewinn pro Akie. Diese Kennzahl wird auch oft für den Entwicklung über die letzten Jahre genutzt. 
        Analysten schätzen meist auch das EPS der nächsten Quartale und des kommenden Jahres.  
        \formulaNote{\formulaText{EPS} = \frac{\formulaText{Net Income}}{\formulaText{Number of Shares}}}    
\end{description}


%
%%
%%%
\section{Statement of Cash Flows}
Dieses Blatt zeigt auf, wie viel Geld tatsächlich in und aus dem Unternehmen geflossen ist.
Hier taucht nur Geld auf, das innerhalb der Finanzperiode tatsächlich in das oder aus dem Unternehmen ging.
So kann z.B. Geld, das ein Kunde erst in der Zukunft bezahlen wird, im Income Statement bereits enthalten sein, im Statement of Cash Flows ist es jedoch nicht eingerechnet, da es das Unternehmen noch nicht erhalten hat.

Einer der wichtigsten Kennwerte, der \freeCashFlow lässt sich aus den hier aufgeführten Zahlen ermitteln.
\formulaNote{\formulaText{Free cash flow} = \formulaText{operating cash flow} - \formulaText{Capital Expenditures}}

%
%%
\subsection{Cash Flow from operating activities}

Dieser Anteil beschreibt das Geld, das das Unternehmen aus seinem Kerngeschäft, z.B. Verkauf von Produkten, tatsächlich eingenommen hat.
Die üblichen Einträge in dieser Auflistung sind nachfolgend beschrieben.
\begin{description}
    \item[Net Income] Nettogewinn. Diese Zahl ist aus dem \englishName{Income Statement} übernommen.
    \item[Depreciation and Amortization] Abschreibungen. Ist ebenfalls aus dem \englishName{Income Statement} übernommen.
    \item[Tax Benefits from Employee Stock Plans] Aktienoptionen für Angestellte des Unternehmens
    \item[Changes in working captal] Veränderung in \englishName{Accounts Receivable}, \englishName{Accounts Payable} und \englishName{Inventories}
    \item[One-Time charges] Ebenfalls aus dem \englishName{Income Statement} übernommen (unter \englishName{Nonrecurring Charges}) 
\end{description}

Es können auch noch mehr oder feiner untergliederte Einträge vorhanden sein.
Die Summe aller aufgelisteten Beträge beschreibt das \textbf{\englishName{Net cash provided by operating activities}} und wird oft auch als \englishName{operating cash flow} bezeichnet.

%
%%
\subsection{Cash Flow from investing activities}

Dieser Anteil beschreibt die Aufwendungen des Unternehmens für Investitionen und Ähnliches.
\begin{description}
    \item[Capital Expenditures] Alle notwendigen Ausgaben um das Geschäft am Laufen zu halten. 
        Hierzu zählen z.B. die Ausgaben für \englishName{PP\&E}.
    \item[Investment Proceeds] Anlagen des Unternehmens  
\end{description}

%
%%
\subsection{Cash Flow from financing activities}

Dieser Abschnitt beinhaltet alle Transaktionen mit den Unternehmenseigner und Kredtigebern.
\begin{description}
    \item[Dividends Paid] Dividendenzahlungen an die Aktionäre
    \item[Issuance/Purchase of Common Stock] Eingenommenes Geld aus der Ausgabe neuer Aktien oder aufgewendetes Geld für Aktienrückkäufe.
    \item[Issuance/Repayments of Debt] Aufnahme von Krediten und Rückzahlungen  
\end{description}
