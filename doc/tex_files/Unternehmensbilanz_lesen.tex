

\chapter{Unternehmensbilanzen lesen}

Die börsennotierten Unternehmen berichten ihre Ergebnisse überlicherweise jedes Quartal  (\englishName{10-Q}) und legen am Ende eines Geschäftsjahres auch eine Jahresbilanz (\englishName{10-K}) vor.
Die wichtigsten Informationen finden sich dabei in drei verschiedenen Blättern, die nachfolgend näher betrachtet werden.

\section{Balance Sheet}
Dieses Blatt gibt einen Überblick über das Vermögen und die Schulden eines Unternehmens und beschreibt den aktuellen finanziellen Zustand der Firma.
Wie der Name andeutet, befinden sich die Zahlen in diesem Blatt im Gleichgewicht (\textit{balance}), da Gleichgewicht
\formulaNote{\formulaText{Assets} - \formulaText{Liablilities} = \formulaText{Equity}}

\subsection{Asset Accounts}
\begin{description}
    \item[Current Assets] sind die Vermögenswerte, die innerhalb eines Finanzzykluses zu Geld gemacht werden.
        Diese umfassen Folgendes 
        \begin{description}
            \item[Cash and Equivalents and short-term-Investments] beschreibt, was das Unternehmen kurzfristig an Geld verfügbar hat.
            \item[Accounts Receivable] umfasst alle Einnahmen, die dem Unternehmen noch geschuldet werden, da beispielsweise die Ware schon an den Kunden geliefert wurde, dessen Zahlung aber noch nicht eingegangen ist.
            \item[Inventories] umfasst Kapital, das in Form von Material etc. vorhanden ist und zur Herstellung der Produkte genutzt wird.
        \end{description}
    \item[Asset Accounts] sind Vermögenswerte, die nicht innerhalb eines Finanzzykluses zu Geld gemacht werden.
        Hierzu zählen Folgende Posten
        \begin{description}
            \item[PP\&E Property, Plant and Equipment] umfasst das Kapital, das in Fabriken, Gebäuden etc. steckt.
            \item[Investments] ist das Kapital, das in Anlagen und Aktien anderer Unternehmen steckt.
            \item[Intangible Assets] beschreibt alle immateriellen Güter. Hierzu zählt z.B. auch der sog. \englishName{Goodwill}. 
                Dies beschreibt das Geld, das bei einer Übernahme des Unternehmens notwendig ist und den Sachwert der Firma übersteigt, da z.B. auch der Ruf und die Marke \glqq gekauft\grqq\ werden.   
        \end{description}
\end{description}

\subsection{Liabilitiy Accounts}

\section{Income Statement}
Dieses Blatt beinhaltet die Einnahmen und Ausgaben innerhalb eines Finanzzykluses und beschreibt somit die aktuelle Entwicklung des Unternehmens.

\begin{description}
    \item[Revenue/Sales]
    \item[Cost of Sales]
    \item[Gross Profit]
    \item[Selling, General and Administrative Expenses (SG\&A)]
    \item[Depreciation and Amortization]
    \item[Nonrecurring Charges/Gains]
    \item[Operating Income]
    \item[Interest Income/Expense]
    \item[Taxes]
    \item[Net Income]
    \item[Number of Shares]
    \item[Earnings per Share]            
\end{description}



\section{Statement of Cash Flows}
Dieses Blatt zeigt auf, wie viel Geld tatsächlich in und aus dem Unternehmen geflossen ist.
Hier taucht nur Geld auf, das innerhalb der Finanzperiode tatsächlich in das oder aus dem Unternehmen ging.
So kann z.B. Geld, das ein Kunde erst in der Zukunft bezahlen wird, im Income Statement bereits enthalten sein, im Statement of Cash Flows ist es jedoch nicht eingerechnet, da es das Unternehmen noch nicht erhalten hat.

\subsection{Cash Flow from operating activities}

\subsection{Cash Flow from investing activities}

\subsection{Cash Flow from financing activities}

